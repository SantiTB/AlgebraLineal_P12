\documentclass{article}
\usepackage[margin=0.5in]{geometry}
\usepackage{amsmath}
\usepackage{graphicx}
\usepackage{amssymb}
\usepackage{multicol}
\usepackage{xcolor}
\usepackage{amsthm}
\usepackage{halloweenmath}
\usepackage{tabto}
\newcommand{\R}{\mathbb{R}}
\newcommand{\A}{\mathcal{A}}
\newenvironment{tightcenter}{%
    \setlength\topsep{0pt}
    \setlength\parskip{0pt}
    \begin{center}
}{%  
    \end{center}
}
\title{Trabajo Práctico 11 - Formas bilineales }
\author{Santiago}
\date{}
\begin{document}
    \maketitle
    \begin{enumerate}
        \item Decir cuáles de las siguientes aplicaciones $\A :\R^2 \times \R^2\to \R$ son bilineales.$\bigskull$
            \begin{enumerate}
                \item $\A((x_1,y_1),(x_2,y_2))=1$\\
                    Por definición, una forma bilineal sobre un $\mathbb{K}$-espacio vectorial $V$(en este caso $\R^2$) es una función $\A:V\times V\to \mathbb{K}$ tal que:
                    \begin{enumerate}
                        \item $\A(\alpha u+v,w)=\alpha \A(u,w)+\A(v,w)$ \quad $ u,v,w\in V,\alpha \in \mathbb{K}$
                        \item $\A(u,\alpha v+w)=\alpha \A(u,v)+\A(u,w)$ \quad $ u,v,w\in V,\alpha \in \mathbb{K}$
                    \end{enumerate}
                    Sean $u,v,w\in \R^2, \alpha \in \R$ tenemos que: $\A(\alpha u+v,w)=1$ por defición, mientras que $\alpha \A(u,w)+\A(v,w)=\alpha \cdot 1 + 1=\alpha +1$, con lo cual no se cumple la primer condición y por ende $\A$ no es una forma bilineal.
                \item $\A((x_1,y_1),(x_2,y_2))=(x_1+y_1)^2-(x_1-y_1)^2$\\
                    Sean $u=(u_1,u_2),v=(v_1,v_2),w=(w_1,w_2)\in \R^2, \alpha \in \R\to \A(\alpha u+v,w)=$
                    \begin{align*}
                        &=(\alpha (u_1,u_2)+(v_1,v_2),(w_1,w_2)) &&\text{Definición $u,v,w$}\\
                        &=((\alpha u_1+v_1,\alpha u_2+v_2),(w_1,w_2)) &&\text{Suma vec y Producto por escalar}\\
                        &=((\alpha u_1+v_1)+(\alpha u_2+v_2))^2-((\alpha u_1+v_1)-(\alpha u_2+v_2))^2 &&\text{Definición } \A\\
                        &=4(\alpha u_1+v_1)(\alpha u_2+v_2) &&\text{Suma en }\R
                    \end{align*}
                    Por otra parte tenemos que $\alpha \A(u,w)+\A(v,w)=$
                    \begin{align*}
                        &=\alpha\A((u_1,u_2),(w_1,w_2))+\A((v_1,v_2),(w_1,w_2))&&\text{Definición }u,v,w\\
                        &=\alpha((u_1+u_2)^2-(u_1-u_2)^2)+(v_1+v_2)^2-(v_1-v_2)^2&&\text{Definición }\A\\
                        &=\alpha(4u_1u_2)+4v_1v_2&&\text{Suma en }\R
                    \end{align*}
                    Como no se cumple la primera condición, $\A$ no es una forma bilineal.
                \item $\A((x_1,y_1),(x_2,y_2))=(x_1-y_1)^2+x_2y_2$\\
                    Sean $u=(u_1,u_2),v=(v_1,v_2),w=(w_1,w_2)\in \R^2, \alpha \in \R\to \A(\alpha u+v,w)=$
                    \begin{align*}
                        &=(\alpha (u_1,u_2)+(v_1,v_2),(w_1,w_2)) &&\text{Definición $u,v,w$}\\
                        &=((\alpha u_1+v_1,\alpha u_2+v_2),(w_1,w_2)) &&\text{Suma vec y Producto por escalar}\\
                        &=((\alpha u_1+v_1)-(\alpha u_2+v_2))^2+w_1w_2 &&\text{Definición } \A\\
                        &=2(\alpha u_1+v_1)(\alpha u_2+v_2)+w_1w_2
                    \end{align*}
                    Por el otro lado $\alpha \A(u,w)+\A(v,w)=$
                    \begin{align*}
                        &=\alpha\A((u_1,u_2),(w_1,w_2))+\A((v_1,v_2),(w_1,w_2))&&\text{Definición }u,v,w\\
                        &=\alpha((u_1-u_2)^2+w_1w_2)+(v_1-v_2)^2+w_1w_2&&\text{Definición }\A
                    \end{align*}
                    En donde se ve que no son iguales, con lo cual $\A$ no es una forma bilineal.
                \item $\A((x_1,y_1),(x_2,y_2))=x_1y_2-x_2y_1$\\
                Sean $u=(u_1,u_2),v=(v_1,v_2),w=(w_1,w_2)\in \R^2, \alpha \in \R\to \A(\alpha u+v,w)=$
                \begin{align*}
                    &=(\alpha (u_1,u_2)+(v_1,v_2),(w_1,w_2)) &&\text{Definición $u,v,w$}\\
                    &=((\alpha u_1+v_1,\alpha u_2+v_2),(w_1,w_2)) &&\text{Suma vec y Producto por escalar}\\
                    &=(\alpha u_1+v_1)w_2+(\alpha u_2+v_2)w_1&&\text{Definición } \A\\
                    &=\alpha(u_1w_2+u_2w_1)+v_1w_2+v_2w_1\\
                    &=\alpha \A(u,w)+\A(v,w)&&\text{Definición }\A
                \end{align*}
                Se cumple la primera condición, ahora para la segunda:
                Sean $u=(u_1,u_2),v=(v_1,v_2),w=(w_1,w_2)\in \R^2, \alpha \in \R\to \A(u,\alpha v+w)=$
                \begin{align*}
                    &=((u_1,u_2),\alpha (v_1,v_2)+(w_1,w_2)) &&\text{Definición $u,v,w$}\\
                    &=((u_1,u_2),(\alpha v_1+w_1,\alpha v_2+w_2)) &&\text{Suma vec y Producto por escalar}\\
                    &=u_1(\alpha v_2+w_2)+u_2(\alpha v_1+w_1)&&\text{Definición } \A\\
                    &=\alpha(u_1v_2+u_2v_1)+u_1w_2+u_2w_1\\
                    &=\alpha \A(u,v)+\A(u,w)&&\text{Definición }\A
                \end{align*}
                Como se cumplen las dos condiciones se puede afirmar que $\A$ es una forma bilineal.
            \end{enumerate}
        \item Sea $A\in\mathbb{K}^{m\times m}$ y sea $\A:\mathbb{K}^{m\times m}\times \mathbb{K}^{m\times m}\to \mathbb{K}$ dada por
            \begin{center}
                $\A(X,Y)=tr(Y^tAX)$\quad para $X,Y\in\mathbb{K}^{m\times m}$
            \end{center}
            Probar que $\A$ es una forma bilineal.\\
            Sea $X,Y,Z\in \mathbb{K}^{m\times m}$ y $\alpha \in \mathbb{K}\to \A(\alpha X+Y,Z)=$
            \begin{align*}
                &=tr(Z^tA(\alpha X+Y))&&\text{Definición }\A\\
                &=tr(Z^tA\alpha X+Z^tAY)&&\text{Distributividad}\\
                &=tr(Z^tA\alpha X)+tr(Z^tAY)&&\text{Propiedades de la traza}\\
                &=\alpha tr(Z^tAX)+tr(Z^tAY)&&\text{Propiedades de la traza}\\
                &=\alpha\A(X,Z)+\A(Y,Z)&&\text{Definición }\A
            \end{align*}
            Cumple la primer condición. Ahora $\A(X,\alpha Y+Z)=$
            \begin{align*}
                &=tr((\alpha Y+Z)^tAX)&&\text{Definición }\A\\
                &=tr(((\alpha Y)^t+Z^t)AX)&&\text{Propiedad de la traspuesta}\\
                &=tr((\alpha Y)^tAX+Z^tAX)&&\text{Distributividad}\\
                &=tr((\alpha Y)^tAX)+tr(Z^tAX)&&\text{Propiedades de la traza}\\
                &=\alpha tr(Y^tAX)+tr(Z^tAX)&&\text{Propiedades de la traza y traspuesta}\\
                &=\alpha\A(X,Y)+\A(X,Z)&&\text{Definición }\A
            \end{align*}
            Al cumplirse las dos condiciones, $\A$ es una forma bilineal.
        \item Sea $V$ un $\mathbb{K}$-espacio vectorial. Probar que $Bil(V)$ es un $\mathbb{K}$-espacio vectorial.
        \item Decir si las siguientes formas bilineales son simétricas:
            \begin{enumerate}
                \item $\A\in Bil(\R^2)$ dada por $\A((x_1,y_1),(x_2,y_2))=3x_1x_2+2x_1y_2+2y_1x_2+y_1y_2$
                    Para que una forma bilineal sea simétrica se debe cumplir $\A(u,v)=\A(v,u)\forall u,v$
                    \begin{align*}
                        \A((x_1,y_1),(x_2,y_2))&=3x_1x_2+2x_1y_2+2y_1x_2+y_1y_2&&\text{Definición}\\
                        &=3x_2x_1+2x_2y_1+2y_2x_1+y_2y_1&&\text{Conmutatividad suma y prod}\\
                        &=\A((x_2,y_2)(x_1,y_1))&&\text{Definición}
                    \end{align*}
                    Por lo tanto, $\A$ es simétrica.
                \item $\A\in Bil(\R^2)$ dada por $\A((x_1,y_1),(x_2,y_2))=x_1x_2+4x_1y_2-3y_1x_2$
                    \begin{align*}
                        \A((x_1,y_1),(x_2,y_2))&=x_1x_2+4x_1y_2+3y_1x_2&&\text{Definición}\\
                        \A((x_2,y_2),(x_1,y_1))&=x_2x_1+4x_2y_1+3y_2x_1&&\text{Definición}\\
                    \end{align*}
                    Se ve que $\A$ no es simétrica.
                \item $\A\in Bil(\R^3)$ dada por $\A((x_1,y_1,z_1),(x_2,y_2,z_2))=x_1x_2+y_1y_2+z_1z_2$
                    \begin{align*}
                        A((x_1,y_1,z_1),(x_2,y_2,z_2))&=x_1x_2+y_1y_2+z_1z_2&&\text{Definición}\\
                        &=x_2x_1+y_2y_1+z_2z_1&&\text{Conmutatividad}\\
                        &=\A((x_2,y_2,z_2),(x_1,y_1,z_1))&&\text{Definición}
                    \end{align*}
                    Por ende, $\A$ es simétrica.
            \end{enumerate}
        \item Para las formas bilineales simétricas del punto anterior, escribir una representación matricial en cualquier base que no sea la canónica del espacio vectorial involucrado. Hallar su rango.
            \begin{itemize}
                \item[(a)] Como estoy trabajando en $\R^2$, tomo de base a $B=\{(1,0),(0,-1)\}$. De la \textbf{Definición 9.6} se tiene que $A_{ij}=\A(b_j,b_i)$ en donde $A=[\A]_B\to$
                    $\begin{cases} 
                        A_{11}=\A((1,0),(1,0))=3 \\
                        A_{12}=\A((0,-1),(1,0))=-2 \\
                        A_{21}=\A((1,0),(0,-1))=-2 \\
                        A_{22}=\A((0,-1),(1,0))=1 \\
                    \end{cases}
                    \to A=\begin{pmatrix}
                        3 & -2 \\
                        -2 & 1 
                    \end{pmatrix}$
                    \quad Se ve que $rg(A)=2$
                \item[(c)] En este caso como base elijo a $B={(1,0,0),(1,1,0),(1,1,1)}$. Usando la misma definición que en el caso anterior:
                    \begin{center}
                        $A_{11}=\A((1,0,0),(1,0,0))=1$\quad$A_{21}=\A((1,0,0),(1,1,0))=1$\quad$A_{31}=\A((1,0,0),(1,1,1))=1$\\
                        $A_{12}=\A((1,1,0),(1,0,0))=1$\quad$A_{22}=\A((1,1,0),(1,1,0))=2$\quad$A_{32}=\A((1,1,0),(1,1,1))=2$\\
                        $A_{13}=\A((1,1,1),(1,0,0))=1$\quad$A_{23}=\A((1,1,1),(1,1,0))=2$\quad$A_{33}=\A((1,1,1),(1,1,1))=3$\\
                        $\to A=\begin{pmatrix}
                            1 & 1 & 1 \\
                            1 & 2 & 2 \\
                            1 & 2 & 3 
                        \end{pmatrix}$\quad $rg(A)=3$
                    \end{center}
            \end{itemize}
        \item Sea $\A\in Bil(V)$ simétrica. Probar que $\A$ es no degenerada si y sólo si $V^\perp = \{0\}$. Recordar que si $S$ es un subconjunto de $V$, entonces
            \begin{tightcenter}    
                $S^\perp = \{v\in V:\A(v,s)=0$\quad$\forall s\in S\}$
            \end{tightcenter}
            $(\Rightarrow):$\\
            $\A\in Bil(V)$ es simétrica y no degenerada $\Rightarrow \forall v\in V, v\neq \vec{0},\exists u\in V/\A(u,v)=\A(v,u)\neq 0\Rightarrow \A(s,v)=0 \Leftrightarrow s=\vec{0}\Rightarrow V^\perp =\{0\}$\\
            $(\Leftarrow):$\\
            $\A\in Bil(V)$ es simétrica y $V^\perp =\{0\}\Rightarrow\A(s,v)=0 \Leftrightarrow s=\vec{0}\Rightarrow \A(u,v)=\A(v,u)=0\Leftrightarrow u=\vec{0}\\\Rightarrow \forall v\in V \exists u\in V-\{0\}/\A(u,v)\neq 0\Rightarrow \A$ es no degenerada.
        \item Sea $\A\in Bil(\R^3)$ dada por
            \begin{center}
                $\A((x_1,y_1,z_1),(x_2,y_2,z_2))=3x_1x_2+4x_1y_2+x_1z_2+4y_1x_1+y_1y_2+6y_1z_2+z_1x_2+2z_1y_2+z_1z_2$
            \end{center}
            \begin{enumerate}
                \item ¿Es simétrica $\A$?¿Es antisimétrica?\\
                    Sea $u=(0,1,0)$ y $v=(0,0,1)\Rightarrow \A(u,v)=6$. Mientras que $\A(v,u)=2$. Se concluye que no es simétrica ni antisimétrica.
                \item Hallar $\A_{sim}$ y $\A_{ant}$ en $Bil(\R^3)$ tales que $\A=\A_{sim}+A_{ant}$\\
                    De la \textbf{Proposición 9.18} y su demostración obtenemos que $\A_{sim}:=\frac{1}{2}(\A+\A^t)$ y $\A_{ant}:=\frac{1}{2}(\A-\A^t)$. Haciendo los cálculos se llega a
                    \begin{tightcenter}
                        $\A_{sim}=3x_1x_2+2x_1y_2+2x_2y_1+x_1z_2+x_2z_1+2x_1y_1+2x_2y_2+y_1y_2+4y_1z_2+4y_2z_1+z_1z_2$\\
                        $\A_{ant}=2x_1y_2-2x_2y_1+2x_1y_1-2x_2y_2+2y_1z_2-2y_2z_1$
                    \end{tightcenter}
            \end{enumerate}
        \item Sea $B=\{u,v\}$ una base de $\R^2$ y sea $\A\in Bil(\R^2)$ dada por
            \begin{center}
                $\A(u,u)=\A(v,v)=0$\quad$\A(u,v)=-\A(v,u)=1$
            \end{center}
            Probar que $\A(x,x)=0$ para todo $x\in V$ y que $\A$ es no degenerada.\\
            La representación matricial de $\A$ en la base $B$ queda determinada por:\\
            $\begin{cases} 
                A_{11}=\A(u,u)=0 \\
                A_{12}=\A(v,u)=-1 \\
                A_{21}=\A(u,v)=1 \\
                A_{22}=\A(v,v)=0 \\
            \end{cases}
            \to A=\begin{pmatrix}
                0 & -1 \\
                1 & 0 
            \end{pmatrix}$
            \quad $rg(A)=2\Rightarrow \A$ es no degenerada.\\
            Usando la ecuación \textbf{9.1}:
            \begin{tightcenter}
                $\A(x,x)=\begin{pmatrix}
                    x & x 
                \end{pmatrix}
                \begin{pmatrix}
                    0 & -1 \\
                    1 & 0 
                \end{pmatrix}
                \begin{pmatrix}
                    x \\
                    x 
                \end{pmatrix}=0$
            \end{tightcenter}
        \item Sea $\A\in Bil(\R^2)$ dada por
            \begin{center}
                $\A((x_1,y_1),(x_2,y_2))=2x_1x_2+x_1y_2+y_1x_2$
            \end{center}
            Hallar su forma cuadrática asociada.\\
            De la \textbf{Definición 9.19} tenemos que
            \begin{tightcenter}
                $Q:V\to \mathbb{K}/Q(v)=\A(v,v)$
            \end{tightcenter}
            es la forma cuadrática asociada a una forma bilineal simétrica. Ya sabemos que es bilineal por el enunciado y además
            \begin{tightcenter}
                $\A((x_1,y_1),(x_2,y_2))=2x_1x_2+x_1y_2+y_1x_2=2x_2x_1+x_2y_1+y_2x_1=\A((x_2,y_2),(x_1,y_1))$\\
                $\to Q(x,y)=2x^2+2xy$
            \end{tightcenter}
        \item Para cada una de las siguientes formas cuadráticas sobre $\R^2$, hallar la forma bilineal simétrica asociada y representarla matricialmente en la base canónica.
            \begin{enumerate}
                \item $Q(x,y)=\alpha x^2$ para $\alpha\in\R$\\
                    De la \textbf{Proposición 9.21} se sabe que
                    \begin{tightcenter}
                        $\A_{sim}(u,v)=\frac{1}{2}(Q(u+v)-Q(u)-Q(v))$
                    \end{tightcenter}
                    \begin{align*}
                        \A_{sim}((x_1,y_1),(x_2,y_2))&=\tfrac{1}{2}[Q(x_1+x_2,y_1+y_2)-Q(x_1,y_1)-Q(x_2,y_2)]\\
                        &=\tfrac{1}{2}[\alpha(x_1+x_2)^2-\alpha x_1^2-\alpha x_2^2]\\
                        &=\alpha x_1x_2\\
                    \end{align*}
                    \begin{tightcenter}
                        $\begin{cases} 
                            A_{11}=\A((1,0),(1,0))=\alpha \\
                            A_{12}=\A((0,1),(1,0))=0 \\
                            A_{21}=\A((1,0),(0,1))=0 \\
                            A_{22}=\A((0,1),(0,1))=0 \\
                        \end{cases}
                        \to A=
                        \begin{pmatrix}
                            \alpha & 0 \\
                            0 & 0 
                        \end{pmatrix}$
                    \end{tightcenter}
                \item $Q(x,y)=\alpha y^2$\\
                    Es un caso muy similar al anterior. Se llega a:
                    \begin{tightcenter}
                        $\A_{sim}((x_1,y_1),(x_2,y_2))=\alpha y_1y_2\quad A=
                        \begin{pmatrix}
                            0 & 0 \\
                            0 & \alpha 
                        \end{pmatrix}
                        $
                    \end{tightcenter}
                \item $Q(x,y)=\alpha xy$ para $\alpha\in\R$
                    \begin{align*}
                        \A_{sim}((x_1,y_1),(x_2,y_2))&=\tfrac{1}{2}[Q(x_1+x_2,y_1+y_2)-Q(x_1,y_1)-Q(x_2,y_2)]\\
                        &=\tfrac{1}{2}[\alpha(x_1+x_2)(y_1+y_2)-\alpha x_1y_1-\alpha x_2y_2]\\
                        &=\tfrac{\alpha}{2}[x_1y_2+x_2y_1]\\
                    \end{align*}
                    \begin{tightcenter}
                        $\begin{cases} 
                            A_{11}=\A((1,0),(1,0))=0 \\
                            A_{12}=\A((0,1),(1,0))=\tfrac{\alpha}{2} \\
                            A_{21}=\A((1,0),(0,1))=\tfrac{\alpha}{2} \\
                            A_{22}=\A((0,1),(0,1))=0 \\
                        \end{cases}
                        \to A=
                        \begin{pmatrix}
                            0 & \tfrac{\alpha}{2} \\
                            \tfrac{\alpha}{2} & 0 
                        \end{pmatrix}$
                    \end{tightcenter}
                \item $Q(x,y)=2x^2-\tfrac{1}{3}xy$\\
                    Con el mismo proceso que los casos anteriores se llega a:
                    \begin{tightcenter}
                        $\A_{sim}((x_1,y_1),(x_2,y_2))=2x_1x_2-\tfrac{1}{6}(x_1y_2+x_2y_1)\quad A=
                        \begin{pmatrix}
                            2 & -\tfrac{1}{6} \\
                            -\tfrac{1}{6} & 0 
                        \end{pmatrix}$
                    \end{tightcenter}
            \end{enumerate}
        \item Probar que:
            \begin{enumerate}
                \item $Q(x,y)=x^2-4xy+5y^2$ es una forma cuadrática definida positiva sobre $\R^2$\\
                    Se debe cumplir que $Q(x,y)>0\quad\forall(x,y)\in\R^2$. Tenemos que $Q(x,y)=x^2-4xy+5y^2=(x-2y)^2+y^2>0\Rightarrow Q$ es definida positiva.
                \item $Q(x,y,z)=7x^2+4xy+y^2-8xz-3z^2$ es una forma cuadrática indefinida sobre $\R^3$\\
                    $\begin{cases} 
                        Q(1,1,1)=7+4+1-8-3=1>0\\
                        Q(1,-2,1)=7-8+4-8-3=-8<0
                    \end{cases}
                    \to Q$ es indefinida.
                \item $Q(x,y,z)=-x^2-2y^2-3z^2$ es una función cuadrática definida negativa sobre $\R^3$\\
                    $Q(x,y,z)=-(x^2+2y^2+3z^2)<0\quad \forall (x,y,z)\in\R^3\Rightarrow Q$ es definida negativa.
            \end{enumerate}
        \item Sea $V$ un $\R$-EV y $Q$ una forma cuadrática sobre $V$ tal que $A$ es la matriz que representa en una base de $B$ (fija) de $V$. Probar que si $\lambda$ es un autovalor de $A$, entonces existe un autovector $v$ asociado a $\lambda$ tal que
            \begin{tightcenter}
                $Q(v)=\lambda\|v\|^2$
            \end{tightcenter}
            Deducir que:
            \begin{enumerate}
                \item Si $Q$ es definida positiva, entonces todos los autovalores de $A$ son positivos.\\
                    Si $Q$ es definida positiva $\Rightarrow \lambda\|v\|^2>0\quad\forall v\neq\vec{0}\Rightarrow \lambda>0$
                \item Si $Q$ es definida negativa, entonces todos los autovalores de $A$ son negativos.\\
                    Si $Q$ es definida negativa $\Rightarrow \lambda\|v\|^2<0\quad\forall v\neq\vec{0}\Rightarrow \lambda<0$
                \item Si $Q$ es indefinido, entonces $A$ tiene autovalores positivos y negativos.
                \item Analizar si vale la recíproca en alguno de los casos anteriores.
            \end{enumerate}
        \item Mostrar que la forma cuadrática en $\R^3$ asociada a la matriz
            \begin{tightcenter}
                $A=
                \begin{pmatrix}
                    -1 & 0 & 0\\
                    0 & -3 & 2\\
                    0 & 2 & -3
                \end{pmatrix}$
            \end{tightcenter}
            es definida negativa.
            \begin{align*}
                Q(v)&=
                \begin{pmatrix}
                    x & y & z    
                \end{pmatrix}
                \begin{pmatrix}
                    -1 & 0 & 0\\
                    0 & -3 & 2\\
                    0 & 2 & -3
                \end{pmatrix}
                \begin{pmatrix}
                    x\\ y\\ z
                \end{pmatrix}\\
                &=
                \begin{pmatrix}
                    x & y & z
                \end{pmatrix}
                \begin{pmatrix}
                    -x \\ -3y +2z \\ 2y -3z
                \end{pmatrix}\\
                &=-x^2-3y^2+2yz+2yz-3z^2\\
                &=-x^2-3y^2+4yz-3z^2\\
                &=-x^2-2(y-z)^2-y^2-z^2\\
                &=-(x^2+2(y-z)^2+y^2+z^2)<0\quad \forall(x,y,z)\in \R^3-\{\vec{0}\}\\
                &\Rightarrow Q \text{ es definida negativa}
            \end{align*}
    \end{enumerate}
\end{document}